
\documentclass[varwidth=15cm,margin=0.07cm]{standalone} 

%%
%% Standard package imports
%% 

\usepackage[T1]{fontenc}             % fontencoding - FIRST load fontencoding 
\usepackage[utf8]{inputenc}          % and only THEN load UTF-8 see https://tex.stackexchange.com/questions/13067/
\usepackage{ucs}                     % unicode support
\usepackage{comment}                 % allows to flag portions of the document using \begin{comment} ... \end{comment}

\usepackage{csquotes}                % quotes
\usepackage{tabto}                   % command \tabto{2cm}
\usepackage[usenames]{color}         % Should load before tcolorbox                  
\usepackage{tcolorbox}
\usepackage{xstring}                 % provides string manipulation, needed for definition of \dref and \durl in TexProcessor.php           


%% minted is included dynamically

%%
%% Delayed minted initialization code
%% * We do not load minted here as part of the normal preamble as it slows down normal latex compilation
%% * we only load minted on demand using attribute minted
%% * However we cannot execute this in situations where we do not load the minted package
%% * Therefore we define a (delayed executing) macro \initializeMinted here
%%   in case the user sets the minted
%%   attribute in the Latex tag - inject this as part of the PHP code.
%% * Still we have the code here and can edit it as part of Dantewiki :-)
%%

% boxsep value sequence is: top left bottom right
\def\initMintedLinenos{
  \tcbuselibrary{breakable}
  \tcbuselibrary{minted}
  \renewcommand\theFancyVerbLine{\footnotesize\arabic{FancyVerbLine}}
  \setminted{linenos=true,numbersep=6pt}
  \BeforeBeginEnvironment{minted}{\begin{tcolorbox}[top=0pt, left=12pt, bottom=0pt, right=0pt,boxrule=0pt]}
  \AfterEndEnvironment{minted}{\end{tcolorbox}}
}

\def\initMintedBox{
  \tcbuselibrary{breakable}
  \tcbuselibrary{minted}
  \BeforeBeginEnvironment{minted}{\begin{tcolorbox}[top=0pt, left=0pt, bottom=0pt, right=0pt,boxrule=1pt]}
  \AfterEndEnvironment{minted}{\end{tcolorbox}}
}


%%
%% DRAWING and GRAPHICS
%%

\usepackage{tikz}
\usetikzlibrary{trees,positioning}
\usetikzlibrary{fpu}
\usetikzlibrary{arrows.meta,calc,decorations.markings,math}
\usetikzlibrary{intersections}
\usetikzlibrary{positioning,fit}%
\usetikzlibrary{quotes,angles}
\usetikzlibrary{mindmap}%  
\usetikzlibrary{shadows}%                           % needed by mindmaps
\usetikzlibrary{automata, arrows}

\usepackage{tikz-qtree}

\usepackage{qrcode}                                   % defines \qrcode[height=1cm,level=L]{ABCD} level: L M Q H for low medium ... quality
\usepackage[percent]{overpic}                         % wrote some text directly over a picture, see 
                                                      % https://tex.stackexchange.com/questions/20792/how-to-superimpose-latex-on-a-picture

\usepackage{pgfplots}
\pgfplotsset{compat=newest}
\pgfplotsset{trig format plots=rad}                   % trigonometry MUST clearly be done in radiants 

\tikzset{trig format=rad}%      pgfplotsset only affects plots, also need this for all of tikz

%%\usepackage{tikz-uml}  %% UNKNOWN 

\def\tikzrad{\pgfplotsset{trig format plots=rad}\tikzset{trig format=rad}}
\def\tikzdeg{\pgfplotsset{trig format plots=deg}\tikzset{trig format=deg}}


\def\tikzAutomaton{
  \tikzdeg%
  \tikzset{
    ->,                                       % makes the edges directed
    >=stealth,                                % makes the arrow heads bold
    node distance=2cm,                        % specifies the minimum distance between two nodes. Change if necessary.
    every state/.style={thick, fill=gray!10}, % sets the properties for each ’state’ node
    initial text=$ $,                         % sets the text that appears on the start arrow
  }%
}



%% TODO: what is this for ??
\newcommand*{\info}[4][16.3]{%
  \node [ annotation, #3, scale=0.65, text width = #1em, inner sep = 2mm ] at (#2) {%
  %\vspace*{-0.5em}%
  \def\BL{\list{$\bullet$}{\topsep=0pt\itemsep=0pt\parsep=0pt\parskip=0pt\labelwidth=8pt\leftmargin=8pt\itemindent=0pt\labelsep=2pt}%
    \def\EL{\endlist}%\baselineskip12pt%
  }%
 \par\nointerlineskip  #4%
  };
}

%%
%% MATHEMATICS
%%
\usepackage[fleqn]{amsmath}          % ams math package with flushed left equations
\usepackage{amssymb,amscd}           % math symbols
\usepackage{mathtools}               % fixes deficiencies in amsmath
\usepackage[all]{xy}
\usepackage{bussproofs}              % proof trees

%%
%% TWEAK amsmath macro to allow \begin{pmatrix}[1.5] for having more space see https://tex.stackexchange.com/questions/14071/how-can-i-increase-the-line-spacing-in-a-matrix
%%
\makeatletter
\renewcommand*\env@matrix[1][\arraystretch]{%
  \edef\arraystretch{#1}%
  \hskip -\arraycolsep
  \let\@ifnextchar\new@ifnextchar
  \array{*\c@MaxMatrixCols c}}
\makeatother

%\input prooftree


%%
%% Include local packages
%%

\input{../local/chc-jpabox.tex}
\input{../local/variants.tex}




\usepackage{ifpdf}%          to be able to detect pdf
\ifpdf{
\pdfcompresslevel=0
\pdfobjcompresslevel=0
%% the following is to get stable builds with the same hash value of the PDF and better control over document privarcy
\pdftrailerid{}             % seed pdf trailer computation with and empty argument for reproducible hash values 
\pdfsuppressptexinfo7       % force pdf driver to omit 1 fullbanner, 2 filename, 4 pagenumber, but leaves info dict intact (for trapped !)
\pdfinfoomitdate1           % force removal of creation and modification dates from PDF file
\pdfpagewidth 15cm
}\else{}\fi

%%
%% Adjusting the lengths
%%

\marginparsep   0cm
\marginparwidth 0cm
\headsep        0cm
\headheight     0cm

\pagestyle{empty}


\setlength{\mathindent}{1cm}
%\makeatletter
%\setlength\@mathmargin{100pt}
%\makeatother


\def\chcdisplayskipadjust{% 
  \setlength{\abovedisplayskip}{4pt}%
  \setlength{\belowdisplayskip}{4pt}%
  \setlength{\abovedisplayshortskip}{4pt}%
  \setlength{\belowdisplayshortskip}{4pt}%
}%

\let\chcoldselectfont\selectfont%
\def\selectfont{%
  \chcoldselectfont%
  \chcdisplayskipadjust%
}%
\chcdisplayskipadjust%


\def\myInitialize{%   needed for initializations to be done after begin{document} which are taken out by the precompilation process
  \parindent 0pt
  \parskip 0.2cm
  \chcdisplayskipadjust
  \initializeVariants
}


%%
%% HYPERREF
%%
\usepackage[hyphens,obeyspaces,spaces]{url}  % must be loaded before hyperref
\usepackage[
  colorlinks,
  hyperfootnotes=true,    % footnote marks as links
  unicode=true,           % https://tex.stackexchange.com/questions/10555/hyperref-warning-token-not-allowed-in-a-pdf-string
  breaklinks=true,         % allow to break links
  hypertexnames=true
]{hyperref}

%%
%% BELOW configures the line breaking algorithm of the url package.
%% DO NOT TOUCH since an additional blank at the wron place may break the algo
%%
\makeatletter
\def\UrlBreaks{\do\.\do\@\do\\\do\/\do\!\do\_\do\|\do\;\do\>\do\]%
  \do\)\do\,\do\?\do\'\do+\do\=\do\#%
  \do A\do B\do C\do D\do E\do F\do G\do H\do I\do J\do K\do L\do M%
  \do N\do O\do P\do Q\do R\do S\do T\do U\do V\do W\do X\do Y\do Z%
  \do a\do b\do c\do d\do e\do f\do g\do h\do i\do j\do k\do l\do m%
  \do n\do o\do p\do q\do r\do s\do t\do u\do v\do w\do x\do y\do z%
  \do 0\do 1\do 2\do 3\do 4\do 5\do 6\do 7\do 8\do 9\do\:%
  \do\.\do\@\do\\\do\/\do\!\do\_\do\|\do\;\do\>\do\]%
  \do\)\do\,\do\?\do\&\do\'\do+\do\=\do\#}%
%\def\UrlBigBreaks{\do\:\do@url@hyp\do\/}%
\def\UrlNoBreaks{}
%\def\UrlOrds{\do\*\do\-\do\~\do\'\do\"\do\-\do\/\do\e}%
\makeatother









\renewcommand{\thempfootnote}{\arabic{mpfootnote}}%    forces footnotes in standalone class to be numeric


%%
%% LOCALIZATION
%%  


%%
%% ENUMERATION and ITEMIZATION
%%

\usepackage[inline]{enumitem}

%% Configuring enumitem
%   CAVE: 1) the \setlist global values for a single list must be defined in a SINGLE \setlist command as they otherwise get overwritten by defaults again
%         2) the settings do not carry over to \newlist defined custom lists
%         3) for the starred inline versions the settings are carried over from the non-starred to the starred versions
%         4) for \newlist defined custom lists the settings do not carry over from the non-starred to the starred versions
%         SIGH !

\setlist{itemjoin={\hspace{0.5cm}}}%                     bit of space as joiner for the inline version

\def\enumerateWithDots{
  \setlist[enumerate]{
    partopsep=1.0\parskip,  % separation added when the preceding is a paragraph
    topsep=-0.5\parskip,    % separation added when the preceding is only a line with no paragraph ending
    itemsep=0cm,
    label=\textbf{(\arabic*)},
    after={\vspace*{1.5\parskip}}
  }
}

\def\enumerateWithBrackets{
  \setlist[enumerate]{
    partopsep=1.0\parskip,  % separation added when the preceding is a paragraph
    topsep=-0.5\parskip,    % separation added when the preceding is only a line with no paragraph ending
    itemsep=0cm,
    label=\textbf{(\arabic*)},
    after={\vspace*{1.5\parskip}}
  }
  \setlist[enumerate,2]{
    label=(\arabic*)
  }
}

\enumerateWithBrackets


%% define enumerated as dotted version should we nevertheless need it.
\newlist{enumerated}{enumerate}{3}\setlist[enumerated]{topsep=0pt,itemsep=0cm,label=\textbf{\arabic*.}}  
\newlist{enumerated*}{enumerate*}{3}\setlist[enumerated*]{topsep=0pt,itemsep=0cm,label=\textbf{\arabic*.}} % need that since for custom lists the * automatism does not work

\setlist[itemize]{
  partopsep=1.0\parskip,
  topsep=-0.5\parskip,
  itemsep=0cm,
  after={\vspace*{1.5\parskip}}
}

\setlist[description]{itemindent=-1.5em, labelsep=0pt,partopsep=\parskip,topsep=-\parskip}



%%
%% TABLES and ARRAYS
%%

\usepackage{array}%                       load the new array package

\arraycolsep=3.0pt%                       adds to the separation of array columns
\def\arraystretch{1.2}%                   increase the distance between rows of an array

\usepackage{adjustbox}%   used for alignment https://tex.stackexchange.com/questions/127056/vertical-alignment-of-graphic-and-tabular

\usepackage{marginnote}
\def\mymarg#1{\hspace*{-6cm}\marginnote{#1}}






%%%%%%%%%%%%%%%%%%%%%%%%%%%%%%%%%%%%%%%%%%%
%% FONTS
%%%%%%%%%%%%%%%%%%%%%%%%%%%%%%%%%%%%%%%%%%%
%\usepackage{concmath}
%\usepackage[OT1]{fontenc}
%\usepackage{txfonts}

\usepackage{stix}                  % needed for \operp and similar symbols
\usepackage{stmaryrd}               % needed for \bigsqcap
\usepackage{fontawesome}           % make fontawesome with some additional symbols available, for example \faTV etc
\usepackage{wasysym}               % some symbols
\usepackage{eurosym}               % euro symbol



%%
%% SHORTCUTS
%%
\def\be{\begin{equation}}
\def\ee{\end{equation}}

% underline boldface
\def\ubf#1{\underline{\textbf{#1}}}


% add shortcut for roman literals
\makeatletter
\newcommand{\rmnum}[1]{\romannumeral #1}
\newcommand{\Rmnum}[1]{\expandafter\@slowromancap\romannumeral #1@}
\makeatother



\def\proof#1#2#3{\prooftree #1 \justifies #2 \thickness=0.04em \using #3 \endprooftree}




\usepackage{soul}                                     % defines strikeout markup   \st{strike out}


%%%
%%% SYMBOLS
%%%



%% MediaWiki:ParsifalInclude/LaTeXSymbols


\newcommand\independent{\protect\mathpalette{\protect\independenT}{\perp}}
\def\independenT#1#2{\mathrel{\rlap{$#1#2$}\mkern2mu{#1#2}}}


\def\vv#1{\overrightarrow{#1}} 
\def\vvplus{\vec{+}}

\def\commutes{\leftrightarrow}



\def\with{\|}

\def\leaves{\spadesuit}                     % Leaves of a tree etc...
\def\interleave{ \mid \mid \mid }   


\def\con{ \mbox{\^{ }}}                     % Concatenation
\def\sts#1{\stackrel{#1}{\longrightarrow}}  % State Transition System
\def\stss#1{\stackrel{#1}{ {\longrightarrow } }_*} % with a star
\def\stsp#1{\stackrel{#1}{ {\longrightarrow } }_+} % with a plus
\def\stsc#1{\stackrel{#1}{ {\longrightarrow } }_\circ} % with a circle
\def\stsh#1{\stackrel{#1}{ {\longrightarrow } }_h} % with a h
\def\pre#1{ {^\bullet #1} }                 % preset in Petri Nets
\def\post#1{ {#1 ^ \bullet} }               % postset in Petri Nets
\def\Me{\mbox{Me}}                          % Me (X,Y) Symbol
\def\bottom{\bot}
\def\with{\mid}

\def\error{\dagger}                          % error symbol error algebras
\def\im{{\cal I}}                            % image of a partial function
\def\nin{\not\in}



\def\negil{\kern-.13em }
\def\negir{\kern-.08em }
\def\LB{[\negil[}
\def\RB{]\negir]}

\def\1{\mbox{{\bf 1}}}
\def\0{\mbox{{\bf 0}}}
\def\T{\mbox{{\bf T}}}
\def\F{\mbox{{\bf F}}}

%%
%% SETS and FUNCTIONS
%%

\def\N{{\Bbb N}}
\def\Q{{\Bbb Q}}
\def\Z{{\Bbb Z}}
\def\K{{\Bbb K}}
\def\R{{\Bbb R}}
\def\C{{\Bbb C}}


%%
%% Specific mathematical arrows
%%

\def\pto{\rightharpoonup}            % partial function:  f: A \pto B
\def\inclusion{\hookrightarrow}      % inclusion, or more general: injective function   f: A \inclusion B

\def\ito{\hookrightarrow}        % inclusion function
\def\pto{\leadsto}               % partially defined function


\def\lins{{\text{\bf Lin}_s}}    % stetige lineare Abbildungen

\def\tr{\text{\bf Tr}}           % spur operator
\def\ker{\text{\bf Kern}}        % Kernel of a linear map
\def\im{\text{\bf Im}}           % Image of a linear map
\def\eig{\text{\bf Eig}}         % Eigenspace of a linear map

\def\norm#1{\| #1 \|}            % norm

%%
%% Pseudocode
%%
 
\def\tIF{\mbox{\tt \ if }}
\def\tTHEN{\mbox{\tt \ then }}
\def\tELSE{\mbox{\tt \ else }}
\def\tFI{\mbox{\tt \ fi }}
\def\tWHILE{\mbox{\tt \ while }}
\def\tDO{\mbox{\tt \ do }}
\def\tREPEAT{\mbox{\tt \ repeat }}
\def\tUNTIL{\mbox{\tt \ until }}

%% 
%% Category Theory
%% 
\def\M{{\frak M}}                       % Morphisms of a category
\def\O{{\frak O}}                       % Objects of a category

\def\codom{\mbox{\textbf{Codom}}}       % Codomain
\def\dom{\mbox{\textbf{Dom}}}           % Domain
\def\Hom{\mbox{\textbf{Hom}}}           % Hom functor

\def\hto{\twoheadrightarrow}
\def\htot{\rightarrowtail}
\def\smallbox{[\hspace*{-0.02cm}]}      % Non-deterministic choice in guarded command language

\def\clolli{{\>\mathbin\multimap\>}}
\def\illolc{{\>\mathbin{\circ\! -}\>}}

\def\before{\mathbin\succeq} 
\def\prebefore{\mathbin{\vec{\succ}}}
\def\extendsto{\mathbin\trianglerighteq}
\def\extendstoim{\mathbin{\vec{\triangleright}}}


\def\trans{\longrightarrow}
\def\transa#1{\stackrel{#1}{\trans}}
\def\transb#1#2{\stackrel{#1}{\trans_#2}}

\def\proof#1#2#3{\prooftree #1 \justifies #2 \thickness=0.04em \using #3 \endprooftree}

\def\entails{\vdash}


\def\tbone#1{\hbox to 5cm{#1}}

\def\THEN{\Rightarrow}
\def\IFF{\Leftrightarrow}



%%%
%%% LEGACY STUFF
%%%
\newcounter{myc}
\newenvironment{mitemize}%
  {\begin{list}{\arabic{myc}.}{\usecounter{myc}\setlength{\topsep}{0pt}}%
    \setlength{\itemsep}{0pt}%
    \setlength{\parskip}{\parskip}}%
  {\end{list}}
  
\newenvironment{menumerate}%
  {\begin{list}{\arabic{myc}.}{\usecounter{myc}\setlength{\topsep}{0pt}}%
    \setlength{\itemsep}{0pt}%
    \setlength{\parskip}{\parskip}}%
  {\end{list}}  
%%%
%%%  END OF TWEAK ITEMIZATIONS
%%%


%%
%% SPACING MACROS
%%

\def\tab{\hspace*{0.7cm}}
\def\n{\break}

\def\up{\vspace*{-\parskip}}          % go up one parskip

\def\mleft{\hspace*{-\mathindent}}    % go left one mathindent space
\def\mright{\hspace*{\mathindent}}    % go right one mathindent space

\def\qquad{\quad\quad}                % double quad
\def\qqquad{\quad\quad\quad}          % triple quad
\def\qqqauad{\quad\quad\quad\quad}    % quadruple quad

%%
%% STRUCTURAL MACROS
%%

\def\title#1{\textbf{\large #1}}


\AtBeginDocument{\parskip 0.2cm}

